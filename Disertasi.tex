%-----------------------------------------------------------------
% Load Proposal Disertasi Class
%-----------------------------------------------------------------
\documentclass[proposal]{disertasiMIPA}
%-----------------------------------------------------------------
% Overwrite package
%-----------------------------------------------------------------
%---pseudocode---%
\usepackage{algorithm}
% \usepackage{algpseudocode}
\usepackage{amssymb}
\usepackage{pdflscape}
% \usepackage[linesnumbered,ruled,vlined]{algorithm2e}
\usepackage{tikz}
\usetikzlibrary{positioning, shapes.geometric}
% \usepackage{algorithmicx}
% \usepackage{algorithm}
% \usepackage{algpseudocode}
\usepackage{amsmath}
% \usepackage{algorithm}
\usepackage{algpseudocode}
% \usepackage{amsmath}
\usepackage{float}

%-----------------%
\usepackage{booktabs}
\usepackage{tabularx}
\usepackage{longtable}
\usepackage{ragged2e}
\usepackage{lipsum}
\usepackage{hyperref}
\sloppy

\hypersetup{
    colorlinks = true,
    citecolor  = {black},
    linkcolor  = {black}
}
\usepackage{pdflscape}
\usepackage{enumitem}
\usepackage{multicol}
\usepackage{array}
\usepackage[margin=1cm]{caption}
\usepackage[margin=1cm]{geometry}
\geometry{
    top=4cm,
    left=4cm,
    right=3cm,
    bottom=3cm
}
%---gambar kamus---%
\usepackage{xcolor}
\usepackage{listings}
\lstset{breaklines=true, basicstyle=\ttfamily}
%--gambar bagan------%

\usepackage{caption}
\usepackage{array}
\usepackage{makecell}

%---gambar kamus---%
\definecolor{codegray}{gray}{0.95}
%--tabel jadwal penelitian---%
\usepackage{amsmath}
\usepackage{enumitem}
\usepackage{graphicx}
\usepackage{placeins} 
\usepackage{float}
\usepackage{makecell}
\usepackage{multirow}
\usepackage{graphicx}
\usepackage{colortbl}
\usepackage[table]{xcolor}
% \usepackage{fancyhdr}
\usepackage[margin=2cm]{geometry}
% \usepackage{algpseudocode}
\usepackage{listings}
\usepackage{xcolor}

% === Algoritma lebih dari 1 halaman === %
\usepackage{caption}


\makeatletter
% Jangan bikin counter baru kalau sudah ada
\@ifundefined{c@algorithm}%
  {\newcounter{algorithm}}% buat hanya jika belum ada
  {}%

% Format nomor algoritma (tanpa per-section)
\renewcommand{\thealgorithm}{\arabic{algorithm}}

% === IEEE-style Algorithm (breakable) with three rules & shared counter ===
\newcommand{\AlgRuleThickness}{0.4pt}
\newcommand{\AlgCaptionTopSep}{.35\baselineskip}
\newcommand{\AlgCaptionBotSep}{.25\baselineskip}
\newcommand{\AlgBodyTopSep}{.30\baselineskip}
\newcommand{\AlgBodyBotSep}{.30\baselineskip}
\newcommand{\AlgBlockVPad}{.50\baselineskip}

\newenvironment{IEEEalgorithm}[1]{%
  \refstepcounter{algorithm}% <-- tambah nomor yang sama dengan float 'algorithm'
  \par\vspace{\AlgBlockVPad}%
  \noindent\hrule height \AlgRuleThickness\relax\par
  \vspace{\AlgCaptionTopSep}%
  \noindent\textbf{Algorithm~\thealgorithm. #1}\par
  \vspace{\AlgCaptionBotSep}%
  \noindent\hrule height \AlgRuleThickness\relax\par
  \vspace{\AlgBodyTopSep}%
  \begin{algorithmic}[1]
}{%
  \end{algorithmic}%
  \vspace{\AlgBodyBotSep}%
  \noindent\hrule height \AlgRuleThickness\relax\par
  \vspace{\AlgBlockVPad}%
}
\makeatother





\lstdefinelanguage{json}{
    basicstyle=\ttfamily\small,
    showstringspaces=false,
    breaklines=true,
    backgroundcolor=\color{codegray},
    literate=
     *{0}{{{\color{blue}0}}}{1}
      {1}{{{\color{blue}1}}}{1}
      {2}{{{\color{blue}2}}}{1}
      {3}{{{\color{blue}3}}}{1}
      {4}{{{\color{blue}4}}}{1}
      {5}{{{\color{blue}5}}}{1}
      {6}{{{\color{blue}6}}}{1}
      {7}{{{\color{blue}7}}}{1}
      {8}{{{\color{blue}8}}}{1}
      {9}{{{\color{blue}9}}}{1}
      {:}{{{\color{black}:}}}{1}
      {,}{{{\color{black},}}}{1}
      {"}{{{\color{red}"}}}{1},
}
%-----------------------------------------------------------------
% Awal Konten Utama Proposal
%-----------------------------------------------------------------
\begin{document}
%-----------------------------------------------------------------
%Awal masukan untuk meta-data proposal
%-----------------------------------------------------------------
\graphicspath{{figures/}}
\titleind{Judul Disertasi}

\titleeng{Disertation Title}

\fullname{Nama Mhs}

\idnum{NIM}

\examdate{16 April 2025}

\degree{Doktor}

\yearsubmit{2025}

\program{Ilmu Komputer}

\headprogram{Dr Wahyono}

\dept{Ilmu Komputer dan Elektronika}

\makeatletter
\def\@firstsupervisor{Nama Promotor}
\def\@secondsupervisor{Nama Co-Promotor}

\DeclareRobustCommand{\firstsupervisor}{\@firstsupervisor}
\DeclareRobustCommand{\secondsupervisor}{\@secondsupervisor}
\makeatother
%\examdate{4 Januari 2002}
%-----------------------------------------------------------------
%Akhir masukan untuk meta-data proposal
%-----------------------------------------------------------------

%-----------------------------------------------------------------
%Awal masukan untuk muka proposal
%-----------------------------------------------------------------
%\cover
\titlepageind 
\approvalpage

\declarepage

%-----------------------------------------------------------------
%Disini awal masukan Acknowledment
%-----------------------------------------------------------------
\acknowledment
\begin{flushright}
	\Large\emph\cal{Karya  ini dipersembahkan \\
		buat Bapak, Ibu, Istri, Anak \\dan Adik tercinta}
\end{flushright}
%-----------------------------------------------------------------
%Disini akhir masukan untuk muka disertasi
%-----------------------------------------------------------------

%-----------------------------------------------------------------
%Disini awal masukan Motto
%-----------------------------------------------------------------
\motto
\emph{Sesungguhnya dalam penciptaan langit dan bumi, dan silih bergantinya
	malam dan siang terdapat tanda-tanda bagi orang-orang yang berakal, (yaitu)
	orang-orang yang mengingat Allah sambil berdiri atau duduk atau dalam keadaan
	berbaring dan mereka memikirkan tentang penciptaan langit dan bumi (seraya
	berkata) : Ya Tuhan kami, tiadalah Engkau menciptakan ini dengan sia-sia, Maha
	Suci Engkau, maka peliharalah kami dari siksa neraka.}

\begin{flushright}
	(Q.S. Ali Imran : 190 - 191)
\end{flushright}

\emph{Maka apabila kamu telah selesai (dari sesuatu urusan), kerjakanlah
	dengan sungguh-sungguh (urusan) yang lain.}

\begin{flushright}
	(Q.S. Alam Nasyrah : 7)
\end{flushright}
%-----------------------------------------------------------------
%Disini akhir masukan untuk Motto
%-----------------------------------------------------------------

%-----------------------------------------------------------------
%Disini awal masukan untuk Prakata
%-----------------------------------------------------------------
\preface
Segala puji dan syukur semata-mata hanya untuk Allah SWT, karena atas segala
rahmat, hidayah dan bantuan-Nya jualah maka akhirnya Proposal Disertasi dengan judul, , ini telah selesai penulis susun.

Telah banyak bantuan yang penulis peroleh selama dalam penulisan Disertasi ini
, untuk itu tak lupa penulis ucapkan terima kasih yang sebesar-besarnya
kepada:
\begin{enumerate}
\item Bapak dan Ibu (Kandung/ Mertua) yang selama ini telah sabar membimbing dan mendoakan penulis tanpa kenal lelah untuk selama-lamanya,
\item Dekan Fakultas Matematika Ilmu Pengetahuan Alam (FMIPA) Universitas Gadjah Mada,
\item{Ketua Departemen Ilmu Komputer dan Elektronika (DIKE), FMIPA Universitas Gadjah Mada, }
\item Pengelola Program Doktor Ilmu Komputer, FMIPA Universitas Gadjah Mada,
\item{Ibu \firstsupervisor{} selaku Pembimbing Utama (Promotor), yang telah
memberikan ilmunya kepada penulis serta dengan penuh kesabaran membimbing penulis,}
\item{Bapak \secondsupervisor{} selaku Pembimbing Pendamping (Ko-promotor) yang telah
memberikan banyak masukan, bimbingan kepada penulis, diskusi dengan penulis, } 

\item{Rekan-rekan angkatan 2024 di jurusan Doktor Ilmu Komputer, Departemen Ilmu Komputer dan Elektronika FMIPA UGM, yang telah
banyak bekerjasama dengan penulis selama belajar di FMIPA UGM,} 
\item{Istri dan Anak saya (Hakim, Azka, Hikam, Rifqy dan Nabila), yang telah memberikan semangat yang kuat dan dorongan untuk terus belajar.} 
\end{enumerate}

Disertasi ini tentunya tidak lepas dari segala kekurangan dan kelemahan, untuk itu
segala kritikan dan saran yang bersifat membangun guna kesempurnaan disertasi ini
sangat diharapkan. Semoga disertasi ini dapat bermanfaat bagi kita semua dan lebih
khusus lagi bagi pengembangan ilmu komputer.

\begin{tabular}{p{7.5cm}c}
	&Yogyakarta, November 2025\\
	&\\
	&\\
	&Penulis
\end{tabular}
\vfill
%-----------------------------------------------------------------
%Disini akhir masukan Prakata
%-----------------------------------------------------------------

\tableofcontents
\listoftables
\listoffigures
%-----------------------------------------------------------------
% Intisari (Bahasa Indonesia)
%-----------------------------------------------------------------
\begin{abstractind}
    Awal abstrak
    
    \bigskip
    \noindent
    \textbf{Kata kunci:} \emph{keyword}, Kata kunci.
\end{abstractind}

%-----------------------------------------------------------------
% Abstract (English)
%-----------------------------------------------------------------
\begin{abstracteng}
    Abstract beginning
    
    \bigskip
    \noindent
    \textbf{Keywords:} \\emph{keyword}, Kata kunci.
\end{abstracteng}

\chapter{PENDAHULUAN}
\pagenumbering{arabic}\setcounter{page}{1}
% \pagenumbering{arabic}\setcounter{page}{1}
\pagestyle{myheadings}

\section{Latar Belakang Masalah}

%\chapter{PENDAHULUAN}
%\section{Latar Belakang Masalah}
%\section{Perumusan Masalah}
%\section{Tujuan Penelitian}
%\section{Batasan Penelitian}
%\section{Manfaat Penelitian}
%\section{Kontribusi Penelitian}

\chapter{TINJAUAN PUSTAKA}

Dalam bab ini disajikan penelitian-penelitian yang sangat berkaitan dan menginspirasi penelitian ini. 
\begin{longtable}{|p{4cm}|p{3cm}|p{3.5cm}|}
    \caption{Studi literatur terkait \emph{Metaheuristik}} 
    \label{tab:tinjauanPustakaMeta} \\
    \hline
    \textbf{Judul} & \textbf{Metode} & \textbf{Hasil} \\
    \hline
    \endfirsthead
    
    \hline
    \multicolumn{3}{|c|}{{\bfseries Lanjutan dari tabel sebelumnya}} \\
    \hline
    \textbf{Judul} & \textbf{Metode} & \textbf{Hasil} \\
    \hline
    \endhead
    
    \hline \multicolumn{3}{|r|}{{Bersambung ke halaman berikutnya}} \\
    \hline
    \endfoot
    \hline
    \endlastfoot
    
    
    Judul& Metodexxx
     &
    hasilxxx\% pada dataset tumor otak\\
    \hline

\end{longtable}

Daftar penelitian sebelumnya yang menggabungkan bla bla.

\chapter{LANDASAN TEORI}

\section{Teori 1}

\begin{figure}
    \centering
    \includegraphics[width=1\linewidth]{figures/dl_skenario.png}
    \caption{Alur Deep Learning}
    \label{fig:deep_learning_process}
\end{figure}


\chapter{METODOLOGI PENELITIAN}

\section{Kerangka Penelitian }
\begin{figure}[H]
    \centering
    \includegraphics[width=1\linewidth]{figures/alur_umum_usulan.png}
    \caption{Kerangka umum penelitian}
    \label{fig:usulan_umum_dasar}
\end{figure}

\subsection{Sub Bab}

\section{Jadwal Rencana Pelaksanaan Penelitian Disertasi}
Pada bagian ini dijelaskan rencana pelaksanaan sejak pengusulan proposal sampai ujian tertutup. Detail jadwal seperti terlihat pada Tabel \ref{tab:rencana-kegiatan-penelitian}. Pada tabel tersebut dijelaskan kegiatan, dan waktu pelaksanaan dalam triwulan (TW). 
\renewcommand{\arraystretch}{1.3} % jarak antar baris

% \begin{landscape}
    \begin{table}[ht!]
    \centering
    \caption{Rencana Kegiatan Penelitian Doktoral}
    \label{tab:rencana-kegiatan-penelitian}
    \begin{tabular}{|c|p{4cm}|*{7}{c|}}
    \hline
    \textbf{No} & \textbf{Kegiatan} & \multicolumn{2}{c|}{2025} & \multicolumn{4}{c|}{2026} & \multicolumn{1}{c|}{2027} \\
    \cline{3-9}
     & & TW 3 & TW 4 & TW 1 & TW 2 & TW 3 & TW 4 & TW 1 \\
    \hline
    1 & Pengajuan Proposal & \cellcolor{gray!50} &  &  &  &  &  &  \\
    \hline
    2 & Seminar Proposal (Seminar kompre) &  & \cellcolor{gray!50} &  &  &  &  &  \\
    \hline
    3 & Perancangan Algoritma xxx &  & \cellcolor{gray!50} &  &  &  &  &  \\
    \hline
    4 & Pengembangan Algoritma xx&  & \cellcolor{gray!50} &  &  &  &  &  \\
    \hline
    5 & Evaluasi Algoritma xxx&  &  & \cellcolor{gray!50} &  &  &  &  \\
    \hline
    6 & Penulisan draft artikel jurnal (Skenario 1)  &  &  & \cellcolor{gray!50} &  \cellcolor{gray!50} &  &  &  \\
    \hline
    7 & Bimbingan dan konsultasi draft artikel jurnal (Skenario 1)  &  &  & \cellcolor{gray!50} &  \cellcolor{gray!50} &  &  &  \\
    \hline
    8 & Submit artikel publikasi jurnal (Skenario 1) &  &  &  & \cellcolor{gray!50} &  &  &  \\
    \hline
    9 & Penulisan artikel publikasi jurnal (Skenario 2) &  &  &  & \cellcolor{gray!50} &  &  &  \\
    \hline
    10 & Bimbingan dan Konsultasi artikel publikasi jurnal (Skenario 2) &  &  &  & \cellcolor{gray!50} &  \cellcolor{gray!50}&  &  \\
    \hline
    11 & Submit artikel publikasi jurnal (FL+GWO) &  &  &  &  & \cellcolor{gray!50} &  &  \\
    \hline
    12 & Bimbingan dan konsultasi draft disertasi &  &  &  &  \cellcolor{gray!50} & \cellcolor{gray!50} & \cellcolor{gray!50} &  \\
    \hline
    13 & Penilaian kelayakan disertasi&  &  &  &  &  & \cellcolor{gray!50} &  \\
    \hline
    14 & Sidang ujian tertutup &  &  &  &  &  &  & \cellcolor{gray!50} \\
    \hline
    \end{tabular}
    \end{table}
    % \end{landscape}

% \input{bab5}
\cleardoublepage
\addcontentsline{toc}{chapter}{DAFTAR PUSTAKA}
%-----------------------------------------------------------------
%Awal masukan untuk Daftar Pustaka
%-----------------------------------------------------------------
\bibliography{pustaka}
\bibliographystyle{pustakastyle}
%-----------------------------------------------------------------
%Akhir masukan Daftar Pustaka
%-----------------------------------------------------------------
%%-----------------------------------------------------------------
%Awal masukan untuk Lampiran A
%-----------------------------------------------------------------
\appendix
\chapter{[CONTOH PERTANYAAN WAWANCARA]} 
\section*{Lembar Wawancara: Evaluasi Kesiapan e-Leadership}

\subsection*{Tujuan}
Mengidentifikasi dan mengevaluasi kesiapan e-leadership berdasarkan empat variabel utama: \textbf{Collaboration}, \textbf{Competence}, \textbf{Direction}, dan \textbf{Culture}.

\subsection*{Informasi Responden}
\begin{itemize}[leftmargin=2cm]
    \item Nama: \underline{\hspace{7cm}}
    \item Jabatan: \underline{\hspace{7cm}}
    \item Institusi/Organisasi: \underline{\hspace{7cm}}
    \item Sektor: 
    \begin{itemize}[label=$\square$]
        \item Akademisi
        \item Pemerintah
        \item Kota
        \item Swasta
    \end{itemize}
    \item Pengalaman dalam Kepemimpinan Digital: \underline{\hspace{2cm}} tahun
\end{itemize}

\subsection*{Pertanyaan Wawancara}

\subsubsection*{I. Collaboration (Kolaborasi)}
1. Sejauh mana Anda dan organisasi Anda mendukung kolaborasi lintas sektor dalam penerapan teknologi digital?  
    \underline{\hspace{3cm}} \\[0.5cm]

2. Apa tantangan utama dalam membangun kolaborasi antara berbagai stakeholder (misalnya, pemerintah, akademisi, sektor swasta)?  
    \underline{\hspace{3cm}} \\[0.5cm]

3. Bagaimana cara Anda memastikan bahwa pemangku kepentingan memiliki peran yang jelas dalam proyek digital?  
    \underline{\hspace{3cm}} \\[0.5cm]

\subsubsection*{II. Competence (Kompetensi)}
4. Bagaimana Anda menilai tingkat kompetensi digital para pemimpin di organisasi Anda?   
    \underline{\hspace{3cm}} \\[0.5cm]

5. Apakah organisasi Anda memiliki program pelatihan khusus untuk meningkatkan kompetensi digital para pemimpin dan staf? Jika ya, mohon jelaskan:  
    \underline{\hspace{3cm}} \\[0.5cm]

6. Apa hambatan terbesar yang Anda hadapi dalam mengembangkan kompetensi digital di organisasi Anda?  
    \underline{\hspace{3cm}} \\[0.5cm]

\subsubsection*{III. Direction (Arah dan Visi)}
7. Sejauh mana organisasi Anda memiliki visi digital yang jelas dan strategis?  Mohon jelaskan bagaimana visi ini diterapkan dalam praktiknya
    \underline{\hspace{3cm}} \\[0.5cm]

8. Bagaimana visi digital organisasi Anda diterjemahkan ke dalam kebijakan atau rencana aksi?  
    \underline{\hspace{3cm}} \\[0.5cm]

9. Bagaimana Anda mengevaluasi kemajuan organisasi Anda dalam mencapai visi digital tersebut?  
    \underline{\hspace{3cm}} \\[0.5cm]

\subsubsection*{IV. Culture (Budaya)}
10. Bagaimana budaya kerja di organisasi Anda mendukung adopsi teknologi digital?  
    \underline{\hspace{3cm}} \\[0.5cm]

11. Apakah terdapat resistensi dari karyawan atau pemimpin dalam menerima perubahan digital? Jika ya, bagaimana Anda menanganinya?  
    \underline{\hspace{3cm}} \\[0.5cm]

12. Bagaimana organisasi Anda memastikan budaya inovasi tetap terjaga di tengah perkembangan teknologi digital?  
    \underline{\hspace{3cm}} \\[0.5cm]

\subsubsection*{Pertanyaan Umum}
13. Menurut Anda, variabel mana di antara Collaboration, Competence, Direction, dan Culture yang paling memengaruhi keberhasilan e-leadership di organisasi Anda? Mohon beri alasan:  
    \underline{\hspace{3cm}} \\[0.5cm]

14. Apakah ada variabel lain yang menurut Anda penting untuk mendukung kesiapan e-leadership? Mohon jelaskan:  
    \underline{\hspace{3cm}} \\[0.5cm]

15. Bagaimana Anda mendefinisikan keberhasilan dalam penerapan kepemimpinan digital di organisasi Anda?  
    \underline{\hspace{3cm}} \\[0.5cm]
\subsection*{Catatan Pewawancara}
\underline{\hspace{3cm}} \\[0.5cm]
\subsection*{Terima Kasih atas Partisipasi Anda}

\chapter{[CONTOH PERTANYAAN KUESIONER]} 
\section*{Kuesioner Penelitian: Kesiapan e-Leadership dalam Kota Cerdas}

\subsection*{Petunjuk Pengisian}
Terima kasih atas partisipasi Anda dalam penelitian ini. Silakan jawab setiap pertanyaan sesuai dengan persepsi Anda menggunakan skala berikut:
\begin{center}
\begin{tabular}{|c|l|}
\hline
\textbf{Skor} & \textbf{Keterangan} \\ \hline
1 & Sangat Tidak Setuju \\ \hline
2 & Tidak Setuju \\ \hline
3 & Netral \\ \hline
4 & Setuju \\ \hline
5 & Sangat Setuju \\ \hline
\end{tabular}
\end{center}

\noindent Mohon lingkari salah satu angka dari 1 sampai 5 yang sesuai dengan pendapat Anda.

---

\subsection*{I. Collaboration (Kolaborasi)}
1. Organisasi saya mendukung kolaborasi lintas sektor (pemerintah, akademisi, swasta) dalam penerapan teknologi digital.  
   \hspace{1cm} 1 \hspace{1cm} 2 \hspace{1cm} 3 \hspace{1cm} 4 \hspace{1cm} 5

2. Kolaborasi antar-divisi di organisasi saya berjalan dengan baik dalam mendukung transformasi digital.  
   \hspace{1cm} 1 \hspace{1cm} 2 \hspace{1cm} 3 \hspace{1cm} 4 \hspace{1cm} 5

3. Organisasi saya menghadapi tantangan dalam membangun kolaborasi dengan stakeholder eksternal.  
   \hspace{1cm} 1 \hspace{1cm} 2 \hspace{1cm} 3 \hspace{1cm} 4 \hspace{1cm} 5

---

\subsection*{II. Competence (Kompetensi)}
4. Pemimpin di organisasi saya memiliki kompetensi digital yang memadai untuk mendukung transformasi digital.  
   \hspace{1cm} 1 \hspace{1cm} 2 \hspace{1cm} 3 \hspace{1cm} 4 \hspace{1cm} 5

5. Organisasi saya secara rutin menyelenggarakan pelatihan untuk meningkatkan kompetensi digital pemimpin dan staf.  
   \hspace{1cm} 1 \hspace{1cm} 2 \hspace{1cm} 3 \hspace{1cm} 4 \hspace{1cm} 5

6. Kompetensi digital pemimpin di organisasi saya membantu dalam pengambilan keputusan berbasis data.  
   \hspace{1cm} 1 \hspace{1cm} 2 \hspace{1cm} 3 \hspace{1cm} 4 \hspace{1cm} 5

---

\subsection*{III. Direction (Arah dan Visi)}
7. Organisasi saya memiliki visi digital yang jelas dan strategis.  
   \hspace{1cm} 1 \hspace{1cm} 2 \hspace{1cm} 3 \hspace{1cm} 4 \hspace{1cm} 5

8. Visi digital organisasi saya diterjemahkan dengan baik ke dalam kebijakan atau rencana aksi.  
   \hspace{1cm} 1 \hspace{1cm} 2 \hspace{1cm} 3 \hspace{1cm} 4 \hspace{1cm} 5

9. Organisasi saya secara rutin mengevaluasi pencapaian visi digital.  
   \hspace{1cm} 1 \hspace{1cm} 2 \hspace{1cm} 3 \hspace{1cm} 4 \hspace{1cm} 5

---

\subsection*{IV. Culture (Budaya)}
10. Budaya kerja di organisasi saya mendukung adopsi teknologi digital.  
    \hspace{1cm} 1 \hspace{1cm} 2 \hspace{1cm} 3 \hspace{1cm} 4 \hspace{1cm} 5

11. Organisasi saya mampu mengatasi resistensi terhadap perubahan digital.  
    \hspace{1cm} 1 \hspace{1cm} 2 \hspace{1cm} 3 \hspace{1cm} 4 \hspace{1cm} 5

12. Organisasi saya memiliki budaya inovasi yang kuat dalam memanfaatkan teknologi digital.  
    \hspace{1cm} 1 \hspace{1cm} 2 \hspace{1cm} 3 \hspace{1cm} 4 \hspace{1cm} 5

---

\subsection*{Pertanyaan Tambahan}
13. Variabel apa yang menurut Anda paling memengaruhi kesiapan e-leadership di organisasi Anda?  
    \underline{\hspace{3cm}}

14. Apakah ada variabel lain yang perlu ditambahkan untuk mendukung kesiapan e-leadership? Mohon jelaskan.  
    \underline{\hspace{3cm}}

15. Bagaimana Anda mendefinisikan keberhasilan e-leadership dalam organisasi Anda?  
    \underline{\hspace{3cm}}

---

\subsection*{Terima Kasih atas Partisipasi Anda}
Mohon kuesioner ini dikembalikan kepada peneliti setelah selesai diisi.

\chapter{[PUBLIKASI REVIEW PAPER 1]}

%-----------------------------------------------------------------
%Akhir masukan Lampiran B
%-----------------------------------------------------------------
%\includepdf[pages=-]{reviewpaper1.pdf}
\end{document}
%-----------------------------------------------------------------
% Akhir Konten Utama Proposal
%-----------------------------------------------------------------