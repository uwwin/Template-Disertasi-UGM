%-----------------------------------------------------------------
%Awal masukan untuk Lampiran A
%-----------------------------------------------------------------
\appendix
\chapter{[CONTOH PERTANYAAN WAWANCARA]} 
\section*{Lembar Wawancara: Evaluasi Kesiapan e-Leadership}

\subsection*{Tujuan}
Mengidentifikasi dan mengevaluasi kesiapan e-leadership berdasarkan empat variabel utama: \textbf{Collaboration}, \textbf{Competence}, \textbf{Direction}, dan \textbf{Culture}.

\subsection*{Informasi Responden}
\begin{itemize}[leftmargin=2cm]
    \item Nama: \underline{\hspace{7cm}}
    \item Jabatan: \underline{\hspace{7cm}}
    \item Institusi/Organisasi: \underline{\hspace{7cm}}
    \item Sektor: 
    \begin{itemize}[label=$\square$]
        \item Akademisi
        \item Pemerintah
        \item Kota
        \item Swasta
    \end{itemize}
    \item Pengalaman dalam Kepemimpinan Digital: \underline{\hspace{2cm}} tahun
\end{itemize}

\subsection*{Pertanyaan Wawancara}

\subsubsection*{I. Collaboration (Kolaborasi)}
1. Sejauh mana Anda dan organisasi Anda mendukung kolaborasi lintas sektor dalam penerapan teknologi digital?  
    \underline{\hspace{3cm}} \\[0.5cm]

2. Apa tantangan utama dalam membangun kolaborasi antara berbagai stakeholder (misalnya, pemerintah, akademisi, sektor swasta)?  
    \underline{\hspace{3cm}} \\[0.5cm]

3. Bagaimana cara Anda memastikan bahwa pemangku kepentingan memiliki peran yang jelas dalam proyek digital?  
    \underline{\hspace{3cm}} \\[0.5cm]

\subsubsection*{II. Competence (Kompetensi)}
4. Bagaimana Anda menilai tingkat kompetensi digital para pemimpin di organisasi Anda?   
    \underline{\hspace{3cm}} \\[0.5cm]

5. Apakah organisasi Anda memiliki program pelatihan khusus untuk meningkatkan kompetensi digital para pemimpin dan staf? Jika ya, mohon jelaskan:  
    \underline{\hspace{3cm}} \\[0.5cm]

6. Apa hambatan terbesar yang Anda hadapi dalam mengembangkan kompetensi digital di organisasi Anda?  
    \underline{\hspace{3cm}} \\[0.5cm]

\subsubsection*{III. Direction (Arah dan Visi)}
7. Sejauh mana organisasi Anda memiliki visi digital yang jelas dan strategis?  Mohon jelaskan bagaimana visi ini diterapkan dalam praktiknya
    \underline{\hspace{3cm}} \\[0.5cm]

8. Bagaimana visi digital organisasi Anda diterjemahkan ke dalam kebijakan atau rencana aksi?  
    \underline{\hspace{3cm}} \\[0.5cm]

9. Bagaimana Anda mengevaluasi kemajuan organisasi Anda dalam mencapai visi digital tersebut?  
    \underline{\hspace{3cm}} \\[0.5cm]

\subsubsection*{IV. Culture (Budaya)}
10. Bagaimana budaya kerja di organisasi Anda mendukung adopsi teknologi digital?  
    \underline{\hspace{3cm}} \\[0.5cm]

11. Apakah terdapat resistensi dari karyawan atau pemimpin dalam menerima perubahan digital? Jika ya, bagaimana Anda menanganinya?  
    \underline{\hspace{3cm}} \\[0.5cm]

12. Bagaimana organisasi Anda memastikan budaya inovasi tetap terjaga di tengah perkembangan teknologi digital?  
    \underline{\hspace{3cm}} \\[0.5cm]

\subsubsection*{Pertanyaan Umum}
13. Menurut Anda, variabel mana di antara Collaboration, Competence, Direction, dan Culture yang paling memengaruhi keberhasilan e-leadership di organisasi Anda? Mohon beri alasan:  
    \underline{\hspace{3cm}} \\[0.5cm]

14. Apakah ada variabel lain yang menurut Anda penting untuk mendukung kesiapan e-leadership? Mohon jelaskan:  
    \underline{\hspace{3cm}} \\[0.5cm]

15. Bagaimana Anda mendefinisikan keberhasilan dalam penerapan kepemimpinan digital di organisasi Anda?  
    \underline{\hspace{3cm}} \\[0.5cm]
\subsection*{Catatan Pewawancara}
\underline{\hspace{3cm}} \\[0.5cm]
\subsection*{Terima Kasih atas Partisipasi Anda}

\chapter{[CONTOH PERTANYAAN KUESIONER]} 
\section*{Kuesioner Penelitian: Kesiapan e-Leadership dalam Kota Cerdas}

\subsection*{Petunjuk Pengisian}
Terima kasih atas partisipasi Anda dalam penelitian ini. Silakan jawab setiap pertanyaan sesuai dengan persepsi Anda menggunakan skala berikut:
\begin{center}
\begin{tabular}{|c|l|}
\hline
\textbf{Skor} & \textbf{Keterangan} \\ \hline
1 & Sangat Tidak Setuju \\ \hline
2 & Tidak Setuju \\ \hline
3 & Netral \\ \hline
4 & Setuju \\ \hline
5 & Sangat Setuju \\ \hline
\end{tabular}
\end{center}

\noindent Mohon lingkari salah satu angka dari 1 sampai 5 yang sesuai dengan pendapat Anda.

---

\subsection*{I. Collaboration (Kolaborasi)}
1. Organisasi saya mendukung kolaborasi lintas sektor (pemerintah, akademisi, swasta) dalam penerapan teknologi digital.  
   \hspace{1cm} 1 \hspace{1cm} 2 \hspace{1cm} 3 \hspace{1cm} 4 \hspace{1cm} 5

2. Kolaborasi antar-divisi di organisasi saya berjalan dengan baik dalam mendukung transformasi digital.  
   \hspace{1cm} 1 \hspace{1cm} 2 \hspace{1cm} 3 \hspace{1cm} 4 \hspace{1cm} 5

3. Organisasi saya menghadapi tantangan dalam membangun kolaborasi dengan stakeholder eksternal.  
   \hspace{1cm} 1 \hspace{1cm} 2 \hspace{1cm} 3 \hspace{1cm} 4 \hspace{1cm} 5

---

\subsection*{II. Competence (Kompetensi)}
4. Pemimpin di organisasi saya memiliki kompetensi digital yang memadai untuk mendukung transformasi digital.  
   \hspace{1cm} 1 \hspace{1cm} 2 \hspace{1cm} 3 \hspace{1cm} 4 \hspace{1cm} 5

5. Organisasi saya secara rutin menyelenggarakan pelatihan untuk meningkatkan kompetensi digital pemimpin dan staf.  
   \hspace{1cm} 1 \hspace{1cm} 2 \hspace{1cm} 3 \hspace{1cm} 4 \hspace{1cm} 5

6. Kompetensi digital pemimpin di organisasi saya membantu dalam pengambilan keputusan berbasis data.  
   \hspace{1cm} 1 \hspace{1cm} 2 \hspace{1cm} 3 \hspace{1cm} 4 \hspace{1cm} 5

---

\subsection*{III. Direction (Arah dan Visi)}
7. Organisasi saya memiliki visi digital yang jelas dan strategis.  
   \hspace{1cm} 1 \hspace{1cm} 2 \hspace{1cm} 3 \hspace{1cm} 4 \hspace{1cm} 5

8. Visi digital organisasi saya diterjemahkan dengan baik ke dalam kebijakan atau rencana aksi.  
   \hspace{1cm} 1 \hspace{1cm} 2 \hspace{1cm} 3 \hspace{1cm} 4 \hspace{1cm} 5

9. Organisasi saya secara rutin mengevaluasi pencapaian visi digital.  
   \hspace{1cm} 1 \hspace{1cm} 2 \hspace{1cm} 3 \hspace{1cm} 4 \hspace{1cm} 5

---

\subsection*{IV. Culture (Budaya)}
10. Budaya kerja di organisasi saya mendukung adopsi teknologi digital.  
    \hspace{1cm} 1 \hspace{1cm} 2 \hspace{1cm} 3 \hspace{1cm} 4 \hspace{1cm} 5

11. Organisasi saya mampu mengatasi resistensi terhadap perubahan digital.  
    \hspace{1cm} 1 \hspace{1cm} 2 \hspace{1cm} 3 \hspace{1cm} 4 \hspace{1cm} 5

12. Organisasi saya memiliki budaya inovasi yang kuat dalam memanfaatkan teknologi digital.  
    \hspace{1cm} 1 \hspace{1cm} 2 \hspace{1cm} 3 \hspace{1cm} 4 \hspace{1cm} 5

---

\subsection*{Pertanyaan Tambahan}
13. Variabel apa yang menurut Anda paling memengaruhi kesiapan e-leadership di organisasi Anda?  
    \underline{\hspace{3cm}}

14. Apakah ada variabel lain yang perlu ditambahkan untuk mendukung kesiapan e-leadership? Mohon jelaskan.  
    \underline{\hspace{3cm}}

15. Bagaimana Anda mendefinisikan keberhasilan e-leadership dalam organisasi Anda?  
    \underline{\hspace{3cm}}

---

\subsection*{Terima Kasih atas Partisipasi Anda}
Mohon kuesioner ini dikembalikan kepada peneliti setelah selesai diisi.

\chapter{[PUBLIKASI REVIEW PAPER 1]}

%-----------------------------------------------------------------
%Akhir masukan Lampiran B
%-----------------------------------------------------------------