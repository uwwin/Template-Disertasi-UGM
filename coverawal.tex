%-----------------------------------------------------------------
%Awal masukan untuk meta-data proposal
%-----------------------------------------------------------------
\graphicspath{{figures/}}
\titleind{Judul Disertasi}

\titleeng{Disertation Title}

\fullname{Nama Mhs}

\idnum{NIM}

\examdate{16 April 2025}

\degree{Doktor}

\yearsubmit{2025}

\program{Ilmu Komputer}

\headprogram{Dr Wahyono}

\dept{Ilmu Komputer dan Elektronika}

\makeatletter
\def\@firstsupervisor{Nama Promotor}
\def\@secondsupervisor{Nama Co-Promotor}

\DeclareRobustCommand{\firstsupervisor}{\@firstsupervisor}
\DeclareRobustCommand{\secondsupervisor}{\@secondsupervisor}
\makeatother
%\examdate{4 Januari 2002}
%-----------------------------------------------------------------
%Akhir masukan untuk meta-data proposal
%-----------------------------------------------------------------

%-----------------------------------------------------------------
%Awal masukan untuk muka proposal
%-----------------------------------------------------------------
%\cover
\titlepageind 
\approvalpage

\declarepage

%-----------------------------------------------------------------
%Disini awal masukan Acknowledment
%-----------------------------------------------------------------
\acknowledment
\begin{flushright}
	\Large\emph\cal{Karya  ini dipersembahkan \\
		buat Bapak, Ibu, Istri, Anak \\dan Adik tercinta}
\end{flushright}
%-----------------------------------------------------------------
%Disini akhir masukan untuk muka disertasi
%-----------------------------------------------------------------

%-----------------------------------------------------------------
%Disini awal masukan Motto
%-----------------------------------------------------------------
\motto
\emph{Sesungguhnya dalam penciptaan langit dan bumi, dan silih bergantinya
	malam dan siang terdapat tanda-tanda bagi orang-orang yang berakal, (yaitu)
	orang-orang yang mengingat Allah sambil berdiri atau duduk atau dalam keadaan
	berbaring dan mereka memikirkan tentang penciptaan langit dan bumi (seraya
	berkata) : Ya Tuhan kami, tiadalah Engkau menciptakan ini dengan sia-sia, Maha
	Suci Engkau, maka peliharalah kami dari siksa neraka.}

\begin{flushright}
	(Q.S. Ali Imran : 190 - 191)
\end{flushright}

\emph{Maka apabila kamu telah selesai (dari sesuatu urusan), kerjakanlah
	dengan sungguh-sungguh (urusan) yang lain.}

\begin{flushright}
	(Q.S. Alam Nasyrah : 7)
\end{flushright}
%-----------------------------------------------------------------
%Disini akhir masukan untuk Motto
%-----------------------------------------------------------------

%-----------------------------------------------------------------
%Disini awal masukan untuk Prakata
%-----------------------------------------------------------------
\preface
Segala puji dan syukur semata-mata hanya untuk Allah SWT, karena atas segala
rahmat, hidayah dan bantuan-Nya jualah maka akhirnya Proposal Disertasi dengan judul, , ini telah selesai penulis susun.

Telah banyak bantuan yang penulis peroleh selama dalam penulisan Disertasi ini
, untuk itu tak lupa penulis ucapkan terima kasih yang sebesar-besarnya
kepada:
\begin{enumerate}
\item Bapak dan Ibu (Kandung/ Mertua) yang selama ini telah sabar membimbing dan mendoakan penulis tanpa kenal lelah untuk selama-lamanya,
\item Dekan Fakultas Matematika Ilmu Pengetahuan Alam (FMIPA) Universitas Gadjah Mada,
\item{Ketua Departemen Ilmu Komputer dan Elektronika (DIKE), FMIPA Universitas Gadjah Mada, }
\item Pengelola Program Doktor Ilmu Komputer, FMIPA Universitas Gadjah Mada,
\item{Ibu \firstsupervisor{} selaku Pembimbing Utama (Promotor), yang telah
memberikan ilmunya kepada penulis serta dengan penuh kesabaran membimbing penulis,}
\item{Bapak \secondsupervisor{} selaku Pembimbing Pendamping (Ko-promotor) yang telah
memberikan banyak masukan, bimbingan kepada penulis, diskusi dengan penulis, } 

\item{Rekan-rekan angkatan 2024 di jurusan Doktor Ilmu Komputer, Departemen Ilmu Komputer dan Elektronika FMIPA UGM, yang telah
banyak bekerjasama dengan penulis selama belajar di FMIPA UGM,} 
\item{Istri dan Anak saya (Hakim, Azka, Hikam, Rifqy dan Nabila), yang telah memberikan semangat yang kuat dan dorongan untuk terus belajar.} 
\end{enumerate}

Disertasi ini tentunya tidak lepas dari segala kekurangan dan kelemahan, untuk itu
segala kritikan dan saran yang bersifat membangun guna kesempurnaan disertasi ini
sangat diharapkan. Semoga disertasi ini dapat bermanfaat bagi kita semua dan lebih
khusus lagi bagi pengembangan ilmu komputer.

\begin{tabular}{p{7.5cm}c}
	&Yogyakarta, November 2025\\
	&\\
	&\\
	&Penulis
\end{tabular}
\vfill
%-----------------------------------------------------------------
%Disini akhir masukan Prakata
%-----------------------------------------------------------------
